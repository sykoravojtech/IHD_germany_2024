% Use this section to briefly summarize the entire text. 
% Highlight limitations and problems, but also make clear statements where they are possible and supported by the analysis. 

In this project, we analyzed cardiovascular diseases with a specific focus on Germany.
We showed that the incidence rate of cardiovascular diseases in Germany is statistically significantly higher than the world average.
We analyzed some of the possible factors that could influence the death rate of ischemic heart disease, which is the most common cardiovascular disease in Germany.


Our model showed that the death rate of ischemic heart disease is influenced by healthcare spending, alcohol consumption, and median age, while 
the effect of fat consumption is mixed and not clear for interpretation. The fact that the effect of fat consumption is not clear is surprising, since it is 
a well-known risk factor for cardiovascular diseases.
We found that lower median age, lower alcohol consumption, and higher healthcare spending
all lead to lower death rates, which was to be expected.


Taking into account Germany's high healthcare spending and acceptable death to incidence ratio for cardiovascular diseases, we do not believe that the cause of the problem lies in the healthcare system.
It appears that it is actually a significant factor to death rate being lower than expected based on lifestyle.
We also found that the German alcohol consumption is higher than the world average (more than two times higher), and that alcohol consumption
in general plays a significant role in this issue. 


The limitation of our analysis is that we didn't analyze all the possible factors that could influence the death rate of ischemic heart disease. 
Such factors could be smoking, vegetable consumption, or physical activity. We focused only on cardiovascular diseases, but it would be interesting to analyze less prevalent diseases as well.
Another possible extension of this project would be to explore if there is any genetic predisposition for cardiovascular diseases in Germany, using a genome-wide association study (GWAS).