% Motivate the problem, situation or topic you decided to
% work on. Describe why it matters (is it of societal, eco-
% nomic, scientific value?). Outline the rest of the paper (use
% references, e.g. to Section 2: What kind of data you are
% working with, how you analyse it, and what kind of conclu-
% sion you reached. The point of the introduction is to make
% the reader want to read the rest of the paper.

% Motivate the problem, situation or topic you decided to work on.
Germany, one of the largest economies in the world with its advanced healthcare 
system, faces an intriguing paradox: its life expectancy lags behind other 
high-income countries. This discrepancy, as highlighted in the analysis by 
\citet{Jasilionis2023} in "\textit{The underwhelming German life expectancy}," 
poses critical questions about the underlying factors contributing to this 
phenomenon. Among these, cardiovascular diseases (CVDs) emerge as a significant area 
of concern. In 2019 CVDs were the leading cause of death in Germany, accounting for 
38\% of all deaths. 

% Describe why it matters (is it of societal, economic, scientific value?).
The reasons behind this phenomenon are still undefined. This paper aims to add to 
the work of \citet{Jasilionis2023} focusing on the most impactful disease out of 
CVDs and investigating how elements such as an aging population, lifestyle choices, 
and dietary habits might correlate with the incidence rate. 

% Outline the rest of the paper (use references, e.g.~to \Cref{sec:methods}: 
% What kind of data you are working with, how you analyse it, and what kind of 
% conclusion you reached. The point of the introduction is to make the reader want 
% to read the rest of the paper.
Firstly, we will provide a brief overview of the most prevalent diseases in Germany 
in the time period from 1990 to 2019. We use the data obtained from the Global 
Burden of Disease study \citep{GBD2019}. We will perform a hypothesis test to 
determine whether the incidence ofCVDs is significantly higher in Germany than the 
global average. After doing so, we introduce the most common CVDs in Germany and 
compare their incidence to the rest of the world. We also offer insight into the 
quality of the healthcare system and its success in treating CVDs.

Secondly, we will investigate the correlation between the incidence of CVDs and 
some of the most common risk factors. In this analysis we used the immense dataset of 
the World Development Incicators by the \citet{worldbank_wdi}. This dataset provided 
us with a wide variety of factors to explore while offering data on the specific 
diseases we were trying to investigate.

Permutation test, correlation testing, multivariate regression